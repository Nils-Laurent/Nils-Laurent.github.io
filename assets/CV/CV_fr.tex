% !TeX spellcheck = fr_FR
\documentclass[a4paper,10pt, french]{article}
\usepackage[french]{babel}
\usepackage[utf8]{inputenc}
% \usepackage[francais]{babel}
\usepackage{ragged2e} % justify, align right ...
\usepackage{multicol} % columns
\usepackage[T1]{fontenc}
\usepackage[dvipsnames, table]{xcolor}
\usepackage[left=1.5cm,right=1.5cm,top=1cm,bottom=1cm]{geometry}
\usepackage{nopageno} % remove page numbers

\usepackage{hyperref}

% bibliography without title
\usepackage{etoolbox}
\patchcmd{\thebibliography}{\section*{\refname}}{}{}{}

%default color
\definecolor{DefaultGray}{rgb}{0.2, 0.2, 0.2}

\definecolor{MyGray}{rgb}{0.4, 0.4, 0.4}
\definecolor{MyLightGray}{rgb}{0.97, 0.97, 0.97}
\definecolor{BorderLightGray}{rgb}{0.83, 0.83, 0.83}

%set default color
\makeatletter
\newcommand{\globalcolor}[1]{%
  \color{#1}\global\let\default@color\current@color
}
\makeatother
\AtBeginDocument{\globalcolor{DefaultGray}}

\begin{document}
\noindent
\textit{\Large \color{Black} Nils Laurent} \hfill {Curriculum vitae (\today)}\\
\vspace{0mm}\\
Né le 7 avril 1994 à Thiers\\
Email : nils.laurent@univ-grenoble-alpes.fr\\
Page web : \url{https://nils-laurent.github.io/}\\
\vspace{5mm}

%---------------------------------------------------------------------
\noindent
\textit{\Large \color{MyGray} \hspace{5mm} Parcours professionel et de formation}
\vspace{2mm}
{\color{DefaultGray}\hrule height .1pt}
\vspace{4mm}

\noindent
\begin{minipage}{0.20\textwidth}
	\color{MyGray} 2022-2023
\end{minipage}
\hfill
\begin{minipage}{0.70\textwidth}
	\textbf{Gipsa-lab} :  Post-doctorat encadré par \textbf{Nicolas Le-Bihan} (Gipsa-lab), \textbf{Salem Said} (LJK) et \textbf{Florent Bouchard} (L2S), dans le cadre du projet \emph{geometrical machine learning: new approaches beyond Riemannian geometry - application to the Stiefel manifold}.
\end{minipage}\\
\vspace{2mm}

\noindent
\begin{minipage}{0.20\textwidth}
  \color{MyGray} 2022
\end{minipage}
\hfill
\begin{minipage}{0.70\textwidth}
  \textbf{LJK, MSTII, Grenoble INP} :  Doctorat dirigé par \textbf{Sylvain Meignen} (LJK), codirigée par \textbf{Bertrand Rivet} (GIPSA-Lab) et \textbf{Julie Fontecave-jallon} (TIMC-IMAG), sur l'\emph{analyse temps-fréquence de signaux multicomposantes bruités}.\\
  La soutenance de thèse s'est déroulée le 9 septembre 2022 devant le jury:
  \begin{multicols}{3}
  	\begin{itemize}
  		\item[] Pierre Chainais
  		\item[] Maria Sandsten
  		\item[] Roland Badeau
  	\end{itemize}
  	\columnbreak
  	\begin{itemize}
  		\item[] Patrick Flandrin
  		\item[] Jérôme Mars
  		\item[] 
  	\end{itemize}
  	\columnbreak
  	\begin{itemize}
  		\item[] Sylvain Meignen
  		\item[] Bertrand Rivet
  		\item[] 
  	\end{itemize}
  \end{multicols}
\end{minipage}\\
\vspace{2mm}

\noindent
\begin{minipage}{0.20\textwidth}
  \color{MyGray} 2019
\end{minipage}
\hfill
\begin{minipage}{0.70\textwidth}
  \textbf{ENSIMAG} : \textbf{Diplôme d'ingénieur} en \textbf{alternance} à l'entreprise Kalray en informatique et mathématiques appliquées, spécialité \emph{modélisation, calcul et simulation}, mention bien.% (15,98/20).
\end{minipage}\\
\vspace{2mm}

\noindent
\begin{minipage}{0.20\textwidth}
  \color{MyGray} 2016
\end{minipage}
\hfill
\begin{minipage}{0.70\textwidth}
  \textbf{IUT Lyon 1} : \textbf{DUT} en informatique.
\end{minipage}\\
\vspace{2mm}

\noindent
\begin{minipage}{0.20\textwidth}
  \color{MyGray} 2014
\end{minipage}
\hfill
\begin{minipage}{0.70\textwidth}
  \textbf{Sundsgården} (Suède) : Études de culture et communication avec obtention du \textbf{prix de camaraderie}.
\end{minipage}\\
\vspace{2mm}

\noindent
\begin{minipage}{0.20\textwidth}
  \color{MyGray} 2013
\end{minipage}
\hfill
\begin{minipage}{0.70\textwidth}
  \textbf{Pierre Desgranges} : \textbf{Baccalauréat} professionnel en systèmes électroniques et numériques, spécialisation en télécommunication et réseaux.
\end{minipage}\\
\vspace{5mm}

%---------------------------------------------------------------------
\noindent
\textit{\Large \color{MyGray} \hspace{5mm} Recherche doctorale}
\vspace{2mm}
{\color{DefaultGray}\hrule height .1pt}
\vspace{4mm}

\textbf{Articles publiés,}
\nocite{*}
\bibliographystyle{unsrt}
\bibliography{refs}
\vspace{5mm}

%\textbf{Articles acceptés,}
%\begin{itemize}
%	\item[*] 
%\end{itemize}
%\vspace{5mm}

\textbf{Articles soumis,}
\begin{itemize}
  \setlength\itemsep{3mm}
%    \item[*] Juan M. Miramont, François Auger, Marcelo A. Colominas, Nils Laurent, and Sylvain Meignen. Unsupervised classification of the spectrogram zeros. \emph{IEEE Transactions on Signal Processing}.
    \item[*] N. Laurent, S. Meignen, M. A. Colominas, J. M. Miramont and F. Auger. A Novel Approach Based on Voronoï cells to Classify Spectrogram Zeros of Multicomponent Signals. \emph{ICASSP 2023}.
\end{itemize}
\vspace{5mm}
\newpage

%---------------------------------------------------------------------
\noindent
\textit{\Large \color{MyGray} \hspace{5mm} Enseignements}
\vspace{2mm}
{\color{DefaultGray}\hrule height .1pt}
\vspace{4mm}

\begin{itemize}
	\setlength\itemsep{3mm}
	\item[*] 2023 [en ce moment] Chargé de TD/TP à l'\textbf{Université Grenoble Alpes}:\\
	Système et environnement de programmation, Bash et C, L1 ($\approx$ 40 hetd).
	\item[*] 2021 Chargé de TD à l'\textbf{Ensimag}:\\
	intégrale de Lebesgue, Fourier, normes et espaces de Banach, L3 ($\approx$ 37 hetd).
	\item[*] 2021 Cours-TD à l'\textbf{Ensimag}:\\
	continuité, développements de Taylor, méthodes numériques, L3 aux alternants ($\approx$ 49 hetd)
	\item[*] 2020 Chargé de TP à l'\textbf{Ensimag}:\\ analyse numérique, L3 ($\approx$ 6 hetd)
	\item[*] 2020 Cours-TD à l'\textbf{Université Grenoble Alpes}:\\
	limites et étude asymptotique, L1 ($\approx$ 22 hetd)
	\item[*] 2020 Chargé de TP à l'\textbf{Université Grenoble Alpes}:\\
	traitement d'images, L1 ($\approx$ 18 hetd)
\end{itemize}

%Enseignements à venir:
%\begin{itemize}
%	\item[*] Système et environnement de programmation (Bash et C) à l'\textbf{UGA} (16.5 HETD de TD et 24 HETD de TP) en L1
%\end{itemize}
%
%Mes enseignements se sont déroulés entre 2020 et 2022, sur le thème de l'analyse. J'en ai effectué à l'Ensimag et à l'Université Grenoble Alpes (UGA). Ils représentent 132,25 \emph{heures équivalent TD} (HETD).
%\vspace{2mm}
%\begin{itemize}
%  \setlength\itemsep{3mm}
%	\item[*] Chargé de TD: Intégration de Lebesgue, Fourier, normes and espaces de Banach à l'\textbf{Ensimag} (37 HETD) en L3 pour les ingénieurs
%	\item[*] Chargé de TP: Analyse numérique à l'\textbf{Ensimag} (6 HETD) en L3 pour les ingénieurs
%	\item[*] Cours-TD: continuité, développements limités, méthodes numériques à l'\textbf{Ensimag} pour les apprentis (48,75 HETD) en L3 pour les ingénieurs en apprentissage
%	\item[*] Cours-TD: limites et analyse asymptotique à l'\textbf{UGA} (22,5 HETD) en L1
%	\item[*] Chargé de TP: traitement d'images à l'\textbf{UGA} (18 HETD) en L1
%\end{itemize}
\vspace{3mm}
Pendant ma thèse, j'ai validé le label \emph{recherche et enseignement supérieur}. Dans ce cadre, j'ai étudié des théories et des méthodes associées à l'enseignement.



\vspace{5mm}

%---------------------------------------------------------------------
\noindent
\textit{\Large \color{MyGray} \hspace{5mm} Autres compétences}
\vspace{2mm}
{\color{DefaultGray}\hrule height .1pt}
\vspace{5mm}

%\noindent
%\begin{minipage}{0.20\textwidth}
%  \color{MyGray} Mathématiques
%\end{minipage}
%\hfill
%\begin{minipage}{0.70\textwidth}
%  \textbf{Mathématiques appliquées} : Traitement d'images, Level sets, Transport optimal, Éléments finis, Calcul des variations, Synthèse d'image et animation, Modélisation surfacique, Assimilation de données, Vision par ordinateur.\\
%\end{minipage}\\
%\vspace{2mm}
%
%\noindent
%\begin{minipage}{0.20\textwidth}
%  \color{MyGray} Informatique
%\end{minipage}
%\hfill
%\begin{minipage}{0.70\textwidth}
%  \textbf{Base théorique} : Analyse automatique de code, Théorie de la complexité, Théorie des langages, Analyse et validation de programmes.\\
%  \textbf{Systèmes} : Systèmes d'exploitation, Calcul sur GPU et HPC, Réseau, Base de données.\\
%  \textbf{Langages} : Matlab, Python, C, LaTeX, HTML, CSS, JavaScript, C++, SQL.\\
%\end{minipage}\\
%\vspace{2mm}

\noindent
\begin{minipage}{0.20\textwidth}
	\color{MyGray} Programmation
\end{minipage}
\hfill
\begin{minipage}{0.70\textwidth}
	Les plus utilisés: Julia, Matlab, C, C++.\\
\end{minipage}\\
\vspace{2mm}

\noindent
\begin{minipage}{0.20\textwidth}
  \color{MyGray} Langues
\end{minipage}
\hfill
\begin{minipage}{0.70\textwidth}
  \textbf{Suédois} : Courant, utilisation régulière avec ma famille suédoise.\\
  \textbf{Anglais} : Langage courant, lecture/écriture d'articles.\\
\end{minipage}\\
\vspace{2mm}

%---------------------------------------------------------------------
\noindent
\textit{\Large \color{MyGray} \hspace{5mm} Services}
\vspace{2mm}
{\color{DefaultGray}\hrule height .1pt}
\vspace{5mm}

\begin{itemize}
  \setlength\itemsep{3mm}
  \item[*] Développement du site web pour l'équipe DAO \url{https://dao.imag.fr/}
  \item[*] Organisateur et animateur d'un événement au lycée Pierre Desgranges pour introduire des méthodologies, concepts théoriques et préparer pour les études supérieures.
\end{itemize}
\vspace{5mm}

% \newpage
%---------------------------------------------------------------------
\noindent
\textit{\Large \color{MyGray} \hspace{5mm} Expérience en Suède}
\vspace{2mm}
{\color{DefaultGray}\hrule height .1pt}
\vspace{5mm}

\noindent
\begin{minipage}{0.20\textwidth}
  \color{MyGray} Cours
\end{minipage}
\hfill
\begin{minipage}{0.70\textwidth}
  Littérature, histoire, mathématiques, anglais, science bilogique et environnementale
\end{minipage}\\
\vspace{2mm}

\noindent
\begin{minipage}{0.20\textwidth}
  \color{MyGray} Apprentissage en autodidacte
\end{minipage}
\hfill
\begin{minipage}{0.70\textwidth}
  Intégration, développement limités, intégration numérique, langages informatiques (C++, python)
\end{minipage}\\
\vspace{2mm}

\paragraph{} Cette expérience m'a permis de prendre du recul pour m'orienter correctement et d'accéder à l'Institut Universitaire de Technologie (IUT) de Lyon1.

\end{document}