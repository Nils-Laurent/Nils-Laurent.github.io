% !TeX spellcheck = fr_FR

{\color{DefaultGray}
	\noindent
	Nils Laurent\\
	Né le 7 avril 1994 à Thiers\\
	Nationalités : français et suédois\\
	Email : nils.laurent@univ-st-etienne.fr\\
	Page web : \url{https://nils-laurent.github.io/}\\
	\vspace{5mm}
	
	%---------------------------------------------------------------------
	\noindent
	\textit{\Large \color{MyGray} \hspace{5mm} Activité professionnelle}
	\vspace{2mm}
	{\color{DefaultGray}\hrule height .1pt}
	\vspace{6mm}
	
	\noindent
	\begin{minipage}{0.10\textwidth}
		\color{MyGray} 2025 --
	\end{minipage}
	\hfill
	\begin{minipage}{0.85\textwidth}
		\textbf{Maitre de conférence} à Roanne rattaché au LASPI et à l'IUT Réseaux \& Télécoms.
	\end{minipage}\\
	\vspace{2mm}
	
	\noindent
	\begin{minipage}{0.10\textwidth}
		\color{MyGray} 2023-2024
	\end{minipage}
	\hfill
	\begin{minipage}{0.85\textwidth}
		\textbf{Post-doctorat} au laboratoire de physique de l'ENS Lyon, encadré par \textbf{Nelly Pustelnik} (LPENSL), \textbf{Julián Tachella} (LPENSL) et \textbf{Elisa Riccietti} (LIP).
	\end{minipage}\\
	\vspace{2mm}
	
	\noindent
	\begin{minipage}{0.10\textwidth}
		\color{MyGray} 2022-2023
	\end{minipage}
	\hfill
	\begin{minipage}{0.85\textwidth}
		\textbf{Post-doctorat} au laboratoire Gipsa-lab, encadré par \textbf{Nicolas Le-Bihan} (Gipsa-lab), \textbf{Salem Said} (LJK) et \textbf{Florent Bouchard} (L2S).
	\end{minipage}\\
	\vspace{2mm}
	
	\noindent
	\begin{minipage}{0.10\textwidth}
		\color{MyGray} 2023
	\end{minipage}
	\hfill
	\begin{minipage}{0.85\textwidth}
		\textbf{Qualification CNU}, sections 26 et 61.
	\end{minipage}\\
	\vspace{2mm}
	
	\noindent
	\begin{minipage}{0.10\textwidth}
		\color{MyGray} 2019-2022
	\end{minipage}
	\hfill
	\begin{minipage}{0.85\textwidth}
		\textbf{Doctorat}, \emph{Analyse temps-fréquence de signaux multicomposantes bruités},\\ Laboratoire LJK dirigé par \textbf{Sylvain Meignen} (LJK), co-dirigé par \textbf{Bertrand Rivet} (GIPSA-Lab) et co-encadré \textbf{Julie Fontecave-jallon} (TIMC-IMAG). Soutenance le 9 septembre 2022:
		\vspace*{1mm}
		
		\begin{itemize}
			\item[] Pierre Chainais -- (président du jury)
			\item[] Maria Sandsten -- (rapportrice)
			\item[] Roland Badeau -- (rapporteur)
			\item[] Patrick Flandrin -- (examinateur)
			\item[] Jérôme Mars -- (examinateur)
		\end{itemize}
	\end{minipage}\\
	\vspace{2mm}
	\vspace{5mm}
	
	%---------------------------------------------------------------------
	\noindent
	\textit{\Large \color{MyGray} \hspace{5mm} Formation}
	\vspace{2mm}
	{\color{DefaultGray}\hrule height .1pt}
	\vspace{6mm}
	
	\begin{itemize}
		\item[*] 2019 \textbf{Diplôme d'ingénieur} à l'Ensimag en alternance, à l'entreprise Kalray.% (15,98/20).
		\item[*] 2016 \textbf{DUT informatique} à l'IUT Lyon 1.
		\item[*] 2014 \textbf{Année d'orientation} Sundsgården (Suède). Obtention du \textbf{prix de camaraderie}.
		\item[*] 2013 \textbf{Baccalauréat} au lycée professionnel Pierre Desgranges.
	\end{itemize}
	\vspace{5mm}
	
	%---------------------------------------------------------------------
	\noindent
	\textit{\Large \color{MyGray} \hspace{5mm} Encadrement}
	\vspace{2mm}
	{\color{DefaultGray}\hrule height .1pt}
	\vspace{6mm}
	
	\textbf{Stage de recherche}
	\vspace{4mm}
	
	\begin{itemize}
		\item[*] Florian VIALLE, \textit{Estimation de la vitesse instantanée sur des signaux harmoniques et applications à la mécanique}, mars-septembre 2025 à l'IUT de Roanne.
	\end{itemize}
	\vspace{5mm}
	
	%---------------------------------------------------------------------
	\noindent
	\textit{\Large \color{MyGray} \hspace{5mm} Expertise}
	\vspace{2mm}
	{\color{DefaultGray}\hrule height .1pt}
	\vspace{6mm}
	
	\textbf{Review, journaux internationaux:} IEEE Trans. on Image Processing, IEEE Trans. on Signal
	Processing
	\vspace{5mm}
	
	\textbf{Review, conférences internationales:} EUSIPCO
	
	\vspace{5mm}
	
	%---------------------------------------------------------------------
	\noindent
	\textit{\Large \color{MyGray} \hspace{5mm} Enseignements}
	\vspace{2mm}
	{\color{DefaultGray}\hrule height .1pt}
	\vspace{4mm}
	
	\begin{itemize}
		\setlength\itemsep{3mm}
		\item[*] 2025-... \textbf{IUT de Roanne}: relatif à la prise de poste de maitre de conférence
		\item[*] 2024-2025 \textbf{IUT de Roanne}: analyse temps-fréquence ($\approx$ 42 hetd)
		\item[*] 2024-2025 \textbf{IUT de Roanne}: diagnostic des systèmes électriques ($\approx$ 66 hetd)
		\item[*] 2024-2025 \textbf{IUT de Roanne}: capteurs et instrumentation ($\approx$ 48 hetd)
		\item[*] 2023 \textbf{Université Grenoble Alpes}: informatique ($\approx$ 40 hetd)
		\item[*] 2021 \textbf{Ensimag}: analyse mathématique, L3 ($\approx$ 37 hetd)
		\item[*] 2021 \textbf{Ensimag}: analyse mathématique pour les alternants ($\approx$ 49 hetd)
		\item[*] 2020 \textbf{Ensimag}: analyse numérique, L3 ($\approx$ 6 hetd)
		\item[*] 2020 \textbf{Université Grenoble Alpes}: analyse mathématique, L1 ($\approx$ 22 hetd)
		\item[*] 2020 \textbf{Université Grenoble Alpes}: traitement d'images, L1 ($\approx$ 18 hetd)
	\end{itemize}
	
	\vspace{5mm}
	
	%---------------------------------------------------------------------
	\noindent
	\textit{\Large \color{MyGray} \hspace{5mm} Publications}
	\vspace{2mm}
	{\color{DefaultGray}\hrule height .1pt}
	\vspace{6mm}
	
	\textbf{Journaux,}
	\begin{enumerate}
		\item \fullcite{bouchard2025beyond}
		\item \fullcite{miramont_unsupervised_2024}
		\item \fullcite{meignen_one_2022}
		\item \fullcite{laurent_local_2022}
		\item \fullcite{laurent_novel_2021-1}
		\item \fullcite{laurent_novel_2020}
	\end{enumerate}
	\vspace{5mm}
	
	\textbf{Conférences,}
	\begin{enumerate}
		\item \fullcite{laurent_estimation_2023}
		\item \fullcite{laurent_novel_2023}
		\item \fullcite{laurent_new_2022}
		\item \fullcite{laurent_novel_2021}
	\end{enumerate}
}% delimiter for \color{DefaultGray}