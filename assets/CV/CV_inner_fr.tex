% !TeX spellcheck = fr_FR

{\color{DefaultGray}
	\noindent
	Nils Laurent\\
	Né le 7 avril 1994 à Thiers\\
	Email : nils.laurent@univ-st-etienne.fr\\
	Page web : \url{https://nils-laurent.github.io/}\\
	Qualifié aux sections CNU 26 et 61\\
	\vspace{5mm}
	
	%---------------------------------------------------------------------
	\noindent
	\textit{\Large \color{MyGray} \hspace{5mm} Parcours professionel et de formation}
	\vspace{2mm}
	{\color{DefaultGray}\hrule height .1pt}
	\vspace{4mm}
	
	\noindent
	\begin{minipage}{0.20\textwidth}
		\color{MyGray} 2024-2025
	\end{minipage}
	\hfill
	\begin{minipage}{0.70\textwidth}
		\emph{Traitement du signal}, \textbf{MCF LRU} au LASPI à Roanne.
	\end{minipage}\\
	\vspace{2mm}
	
	\noindent
	\begin{minipage}{0.20\textwidth}
		\color{MyGray} 2023-2024
	\end{minipage}
	\hfill
	\begin{minipage}{0.70\textwidth}
		\emph{Etudes de problèmes inverses multiniveaux par algorithmes déroulés et ``Plug and Play''}, \textbf{post-doctorat} au laboratoire de physique de l'ENS Lyon, encadré par \textbf{Nelly Pustelnik} (LPENSL), \textbf{Julián Tachella} (LPENSL) et \textbf{Elisa Riccietti} (LIP).
	\end{minipage}\\
	\vspace{2mm}
	
	\noindent
	\begin{minipage}{0.20\textwidth}
		\color{MyGray} 2022-2023
	\end{minipage}
	\hfill
	\begin{minipage}{0.70\textwidth}
		\emph{Estimation efficace de la moyenne Riemannienne -- application aux variétés de Stiefel et Grassmann}, \textbf{post-doctorat} au laboratoire Gipsa-lab, encadré par \textbf{Nicolas Le-Bihan} (Gipsa-lab), \textbf{Salem Said} (LJK) et \textbf{Florent Bouchard} (L2S).
	\end{minipage}\\
	\vspace{2mm}
	
	\noindent
	\begin{minipage}{0.20\textwidth}
		\color{MyGray} 2019-2022
	\end{minipage}
	\hfill
	\begin{minipage}{0.70\textwidth}
		\emph{Analyse temps-fréquence de signaux multicomposantes bruités},\\ \textbf{doctorat} au  laboratoire LJK dirigé par \textbf{Sylvain Meignen} (LJK), co-dirigé par \textbf{Bertrand Rivet} (GIPSA-Lab) et co-encadré \textbf{Julie Fontecave-jallon} (TIMC-IMAG).\\
		La soutenance s'est déroulée le 9 septembre 2022 devant le jury:
		\begin{itemize}
			\item[] Pierre Chainais -- (président du jury)
			\item[] Maria Sandsten -- (rapportrice)
			\item[] Roland Badeau -- (rapporteur)
			\item[] Patrick Flandrin -- (examinateur)
			\item[] Jérôme Mars -- (examinateur)
			\item[] Sylvain Meignen -- (directeur)
			\item[] Bertrand Rivet -- (co-directeur)
		\end{itemize}
	\end{minipage}\\
	\vspace{2mm}
	
	\noindent
	\begin{minipage}{0.20\textwidth}
		\color{MyGray} 2019
	\end{minipage}
	\hfill
	\begin{minipage}{0.70\textwidth}
		\textbf{Diplôme d'ingénieur} à l'\textbf{Ensimag}, filière modélisation mathématiques, mention bien. Alternance effectuée à l'entreprise Kalray.% (15,98/20).
	\end{minipage}\\
	\vspace{2mm}
	
	\noindent
	\begin{minipage}{0.20\textwidth}
		\color{MyGray} 2016
	\end{minipage}
	\hfill
	\begin{minipage}{0.70\textwidth}
		\textbf{DUT informatique} à l'IUT Lyon 1.
	\end{minipage}\\
	\vspace{2mm}
	
	\noindent
	\begin{minipage}{0.20\textwidth}
		\color{MyGray} 2014
	\end{minipage}
	\hfill
	\begin{minipage}{0.70\textwidth}
		Culture et communication à \textbf{Sundsgården} en Suède, et obtention du \textbf{prix de camaraderie}.
	\end{minipage}\\
	\vspace{2mm}
	
	\noindent
	\begin{minipage}{0.20\textwidth}
		\color{MyGray} 2013
	\end{minipage}
	\hfill
	\begin{minipage}{0.70\textwidth}
		\textbf{Baccalauréat} au lycée professionnel Pierre Desgranges: systèmes électroniques et numériques, spécialisation en télécommunication et réseaux.
	\end{minipage}\\
	\vspace{5mm}
	
	%---------------------------------------------------------------------
	\noindent
	\textit{\Large \color{MyGray} \hspace{5mm} Travaux de recherche}
	\vspace{2mm}
	{\color{DefaultGray}\hrule height .1pt}
	\vspace{4mm}
	
	\vspace{5mm}
	
	\textbf{Article accepté,} \fullcite{bouchard_beyond_2025}, \textit{IEEE Signal Processing Letters}
	
	\vspace{5mm}
	
	\textbf{Articles soumis,}
	\begin{enumerate}
		\item \fullcite{laurent_multilevel_2025}, \textit{IEEE Transactions on Computational Imaging}
		\item \fullcite{polisano_disentangling_2025}, \textit{Eusipco 2025}
		\item Une variante de l'article \fullcite{laurent_multilevel_2025} a été soumise au Gretsi 2025 (disponible sur HAL prochainement).
	\end{enumerate}
	
	\textbf{Articles publiés,}
	\begin{enumerate}
		\item \fullcite{laurent_estimation_2023}
		\item \fullcite{miramont_unsupervised_2024}
		\item \fullcite{laurent_novel_2023}
		\item \fullcite{meignen_one_2022}
		\item \fullcite{laurent_new_2022}
		\item \fullcite{laurent_local_2022}
		\item \fullcite{laurent_novel_2021}
		\item \fullcite{laurent_novel_2021-1}
		\item \fullcite{laurent_novel_2020}
	\end{enumerate}
	
	\vspace{5mm}
	
%	\textbf{Articles en préparation,}
%	\begin{itemize}
%		\setlength\itemsep{3mm}
%		\item[*] Juan M. Miramont, François Auger, Marcelo A. Colominas, Nils Laurent, and Sylvain Meignen. Unsupervised classification of the spectrogram zeros.
%		\item[*] Nils Laurent , Nicolas Le Bihan , Florent Bouchard, Salem Said. Projection based estimation of the barycenter on the Stiefel manifold.
%	\end{itemize}
%	\vspace{5mm}
	
	%---------------------------------------------------------------------
	\noindent
	\textit{\Large \color{MyGray} \hspace{5mm} Enseignements}
	\vspace{2mm}
	{\color{DefaultGray}\hrule height .1pt}
	\vspace{4mm}
	
	\begin{itemize}
		\setlength\itemsep{3mm}
		\item[*] 2023 Chargé de TD/TP à l'\textbf{Université Grenoble Alpes}:\\
		Système et environnement de programmation, Bash et C, L1 ($\approx$ 40 hetd)
		\item[*] 2021 Chargé de TD à l'\textbf{Ensimag}:\\
		intégrale de Lebesgue, Fourier, normes et espaces de Banach, L3 ($\approx$ 37 hetd)
		\item[*] 2021 Cours-TD à l'\textbf{Ensimag}:\\
		continuité, développements de Taylor, méthodes numériques, L3 aux alternants ($\approx$ 49 hetd)
		\item[*] 2020 Chargé de TP à l'\textbf{Ensimag}:\\ analyse numérique, L3 ($\approx$ 6 hetd)
		\item[*] 2020 Cours-TD à l'\textbf{Université Grenoble Alpes}:\\
		limites et étude asymptotique, L1 ($\approx$ 22 hetd)
		\item[*] 2020 Chargé de TP à l'\textbf{Université Grenoble Alpes}:\\
		traitement d'images, L1 ($\approx$ 18 hetd)
	\end{itemize}
	
	\vspace{5mm}
	
	%---------------------------------------------------------------------
	\noindent
	\textit{\Large \color{MyGray} \hspace{5mm} Label recherche et enseignement supérieur}
	\vspace{2mm}
	{\color{DefaultGray}\hrule height .1pt}
	\vspace{5mm}
%	Pendant ma thèse, j'ai validé le label \emph{recherche et enseignement supérieur}. Dans ce cadre, j'ai étudié des théories et des méthodes associées à la pédagogie et à l'enseignement. Suite à cette formation, j'ai construit un portfolio qui est disponible en pièce jointe, où je détaille un processus réflexif par rapports à mes précédents enseignements.
	Le label de thèse \emph{recherche et enseignement supérieur} que j'ai pu valider dans le cadre de ma thèse a pour but de former les doctorants sur la pédagogie. Il traite par exemple
	\begin{itemize}
		\item[*] la construction d'une grille critèriée qui peut être utilisée en amont ou au début du cours, notamment pour présenter les objectifs du cours et identifier les obstacles épistémologiques;
		\item[*] différentes méthodes pour l'enseignement, comme l'approche par compétences, l'apprentissage par projet et la classe inversée;
		\item[*] la création d'un \emph{portfolio}, dans lequel on décrit le processus réflexif sur nos enseignements que nous avons mis en place.
	\end{itemize}
	
	\vspace{5mm}
	
	%---------------------------------------------------------------------
	\noindent
	\textit{\Large \color{MyGray} \hspace{5mm} Autres compétences}
	\vspace{2mm}
	{\color{DefaultGray}\hrule height .1pt}
	\vspace{5mm}
	
	\noindent
	\begin{minipage}{0.20\textwidth}
		\color{MyGray} Programmation
	\end{minipage}
	\hfill
	\begin{minipage}{0.70\textwidth}
		Les plus utilisés: Python, Matlab, C, C++, Julia.\\
	\end{minipage}\\
	\vspace{2mm}
	
	\noindent
	\begin{minipage}{0.20\textwidth}
		\color{MyGray} Langues
	\end{minipage}
	\hfill
	\begin{minipage}{0.70\textwidth}
		\textbf{Suédois} : Courant, utilisation régulière avec ma famille suédoise.\\
		\textbf{Anglais} : Langage courant, lecture/écriture d'articles.\\
	\end{minipage}\\
	\vspace{2mm}
	
	%---------------------------------------------------------------------
	\noindent
	\textit{\Large \color{MyGray} \hspace{5mm} Services}
	\vspace{2mm}
	{\color{DefaultGray}\hrule height .1pt}
	\vspace{5mm}
	
	\begin{itemize}
		\setlength\itemsep{3mm}
		\item[*] Développement du site web pour l'équipe DAO \url{https://dao.imag.fr/}
		\item[*] Organisateur et animateur d'un événement au lycée Pierre Desgranges pour introduire des méthodologies, concepts théoriques et préparer pour les études supérieures. Cet événement est décrit en détails sur mon site web.
		\item[*] Deux reviews pour des journaux de traitement du signal.
	\end{itemize}
	\vspace{5mm}
	
	%---------------------------------------------------------------------
	\noindent
	\textit{\Large \color{MyGray} \hspace{5mm} Expérience en Suède}
	\vspace{2mm}
	{\color{DefaultGray}\hrule height .1pt}
	\vspace{5mm}
	
	\noindent
	\begin{minipage}{0.20\textwidth}
		\color{MyGray} Cours
	\end{minipage}
	\hfill
	\begin{minipage}{0.70\textwidth}
		Littérature, histoire, mathématiques, anglais, science bilogique et environnementale
	\end{minipage}\\
	\vspace{2mm}
	
	\noindent
	\begin{minipage}{0.20\textwidth}
		\color{MyGray} Apprentissage en autodidacte
	\end{minipage}
	\hfill
	\begin{minipage}{0.70\textwidth}
		Intégration, développement limités, intégration numérique, langages informatiques (C++, python)
	\end{minipage}\\
	\vspace{2mm}
	
	\paragraph{} Cette expérience m'a permis de prendre du recul pour m'orienter correctement et d'accéder à l'IUT de Lyon1.
	
	\paragraph{} Le \emph{prix de camaraderie} est décerné par l'école à deux des élèves en fin d'année scolaire. Le document est ajouté en pièce jointe. Il n'est pas traduit pour éviter de déformer les propos qui y sont inscrits.
}